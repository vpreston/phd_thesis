\def\blue#1{\textcolor{blue}{{#1}}}
\newcommand{\VP}[1]{\textit{\color{blue} VP: #1}}
\newcommand{\td}[1]{\textbf{\color{red} [TODO: #1]}}

% -----------------------------------------------------------------------------
% POMDP models
% -----------------------------------------------------------------------------
\newcommand{\A}{\mathcal{A}} % set of actions
\newcommand{\Ss}{\mathcal{S}} % set of states
\newcommand{\Zz}{\mathcal{Z}} % set of observations

% -----------------------------------------------------------------------------
% PHUMES
% -----------------------------------------------------------------------------
\newcommand{\x}{\mathbf{x}} % the unknown parameters
\newcommand{\f}{f} % the numerical model

% -----------------------------------------------------------------------------
% Stylized Acronyms
% -----------------------------------------------------------------------------
\newcommand{\PHORTEX}{\textsc{Phortex}\xspace}
\newcommand{\phortex}{\textbf{PH}ysically-informed \textbf{O}perational \textbf{R}obotic \textbf{T}rajectories for \textbf{EX}peditions\xspace}
\newcommand{\PHUMES}{\textsc{Phumes}\xspace}
\newcommand{\phumes}{\textbf{PH}ysically-informed \textbf{U}ncertainty \textbf{M}odels for \textbf{E}nvironment \textbf{S}patiotemporality}
\newcommand{\Sentry}{\emph{Sentry}\xspace}
\newcommand{\POMDP}{\textcs{Pomdp}\xspace}

%-------------------------------------------------------------------------------------------------------------------------------------------------------------------------------------------------------------------
% Mathematical sets
%-------------------------------------------------------------------------------------------------------------------------------------------------------------------------------------------------------------------
\def\reals{\mathbb{R}} % Real number symbol
\def\R{\mathbb{R}}
\def\integers{\mathbb{Z}} % Integer symbol
\def\Z{\mathbb{Z}}
\def\rationals{\mathbb{Q}} % Rational numbers
\def\Q{\mathbb{Q}}
\def\naturals{\mathbb{N}} % Natural numbers
\def\N{\mathbb{N}}
\def\complex{\mathbb{C}} % Complex numbers

%-------------------------------------------------------------------------------------------------------------------------------------------------------------------------------------------------------------------
% Special symbols
%-------------------------------------------------------------------------------------------------------------------------------------------------------------------------------------------------------------------
\def\<{\left\langle} % Angle brackets
\def\>{\right\rangle}

\def\iff{\Leftrightarrow}
%\def\choose#1#2{\left(\begin{array}{c}{#1} \\ {#2}\end{array}\right)}
\def\chooses#1#2{{}_{#1}C_{#2}}
\def\defeq{\triangleq} % defined equal to
\newcommand{\bs}{\backslash} % backslash
\def\half{\frac{1}{2}}
\def\textint{{\textstyle\int}} % Sum in textstyle form
\def\texthalf{{\textstyle\frac{1}{2}}}
\newcommand{\textfrac}[2]{{\textstyle\frac{#1}{#2}}}

% Semidefinite orders
\newcommand{\psdle}{\preccurlyeq}
\newcommand{\psdge}{\succcurlyeq}
\newcommand{\psdlt}{\prec}
\newcommand{\psdgt}{\succ}

%-------------------------------------------------------------------------------------------------------------------------------------------------------------------------------------------------------------------
% Vectors and matrices
%-------------------------------------------------------------------------------------------------------------------------------------------------------------------------------------------------------------------
\newcommand{\boldone}{\mbf{1}} % Bold 1
\newcommand{\boldzero}{\mbf{0}} % Bold 1
\def\v#1{\mbi{#1}} % Vector notation
\newcommand{\norm}[1]{\left\|{#1}\right\|} % A norm with 1 argument
\newcommand{\onenorm}[1]{\norm{#1}_1} % L1 norm
\newcommand{\twonorm}[1]{\norm{#1}_2} % L2 norm
\newcommand{\infnorm}[1]{\norm{#1}_{\infty}} % Linfty norm
\newcommand{\dualnorm}[1]{\norm{#1}_{*}} % Dual norm
\newcommand{\pnorm}[1]{\norm{#1}_{p}} % p norm
\newcommand{\qnorm}[1]{\norm{#1}_{q}} % Dual norm for p norm
\newcommand{\opnorm}[1]{\norm{#1}_{\mathrm{op}}} % Operator norm
\newcommand{\fronorm}[1]{\norm{#1}_{F}} % Frobenius norm
\def\staticnorm#1{\|{#1}\|} % A static norm that does not resize with input
\newcommand{\statictwonorm}[1]{\staticnorm{#1}_2} % L2 norm
\newcommand{\inner}[2]{\langle{#1},{#2}\rangle} % Inner product
\newcommand{\binner}[2]{\left\langle{#1},{#2}\right\rangle} % Inner product with
\newcommand{\locnorm}[2]{\norm{#1}_{#2}} % Inner product
\def\what#1{\widehat{#1}}

\def\twovec#1#2{\left[\begin{array}{c}{#1} \\ {#2}\end{array}\right]}
\def\threevec#1#2#3{\left[\begin{array}{c}{#1} \\ {#2} \\ {#3} \end{array}\right]}
\def\nvec#1#2#3{\left[\begin{array}{c}{#1} \\ {#2} \\ \vdots \\ {#3}\end{array}\right]} % An n-vector with three arguments

% ------------------------------------------------------------------------
% Eigenvalues
% ------------------------------------------------------------------------
\def\maxeig#1{\lambda_{\mathrm{max}}\left({#1}\right)}
\def\mineig#1{\lambda_{\mathrm{min}}\left({#1}\right)}

%-------------------------------------------------------------------------------------------------------------------------------------------------------------------------------------------------------------------
% Operators
%-------------------------------------------------------------------------------------------------------------------------------------------------------------------------------------------------------------------
\def\Re{\operatorname{Re}} % Real part
\def\indic#1{\mathbb{I}\left[{#1}\right]} % Indicator function
\def\staticindic#1{\mathbb{I}[{#1}]} % Indicator function
\def\logarg#1{\log\left({#1}\right)} % log with argument
\def\polylog{\operatorname{polylog}}
\def\maxarg#1{\max\left({#1}\right)} % max with argument
\def\minarg#1{\min\left({#1}\right)} % min with argument
\def\E{\mathbb{E}} % Expectation symbol
\def\Earg#1{\E\left[{#1}\right]}
\def\Esub#1{\E_{#1}}
\def\Esubarg#1#2{\E_{#1}\left[{#2}\right]}
\def\bigO#1{\mathcal{O}(#1)} % big-oh notation
\def\littleO#1{o(#1)} % big-oh notation
\def\P{\mathbb{P}} % Probability symbol
\def\Parg#1{\P\left({#1}\right)}
\def\Psubarg#1#2{\P_{#1}\left[{#2}\right]}
\def\Trarg#1{\Tr\left[{#1}\right]} % Trace with argument
\def\trarg#1{\tr\left[{#1}\right]} % trace with argument
\def\Var{\mrm{Var}} % Variance symbol
\def\Vararg#1{\Var\left[{#1}\right]}
\def\Varsubarg#1#2{\Var_{#1}\left[{#2}\right]}
\def\Cov{\mrm{Cov}} % Covariance symbol
\def\Covarg#1{\Cov\left[{#1}\right]}
\def\Covsubarg#1#2{\Cov_{#1}\left[{#2}\right]}
\def\Corr{\mrm{Corr}} % Covariance symbol
\def\Corrarg#1{\Corr\left[{#1}\right]}
\def\Corrsubarg#1#2{\Corr_{#1}\left[{#2}\right]}
\newcommand{\info}[3][{}]{\mathbb{I}_{#1}\left({#2};{#3}\right)} % Information symbol
%\renewcommand{\exp}[1]{\operatorname{exp}\left(#1\right)} % Exponential
\newcommand{\staticexp}[1]{\operatorname{exp}(#1)} % An exponential with parens that do not resize with input
\newcommand{\loglihood}[0]{\mathcal{L}} % log likelihood
\DeclareMathOperator*{\truemax}{max} % Jan Hlavacek
\newcommand{\overbar}[1]{\mkern 1.5mu\overline{\mkern-1.5mu#1\mkern-1.5mu}\mkern 1.5mu}

%-------------------------------------------------------------------------------------------------------------------------------------------------------------------------------------------------------------------
% Distributions
%-------------------------------------------------------------------------------------------------------------------------------------------------------------------------------------------------------------------
\newcommand{\Gsn}{\mathcal{N}}
\newcommand{\BeP}{\textnormal{BeP}}
\newcommand{\Ber}{\textnormal{Ber}}
\newcommand{\Bern}{\textnormal{Bern}}
\newcommand{\Bet}{\textnormal{Beta}}
\newcommand{\Bin}{\textnormal{Bin}}
\newcommand{\BP}{\textnormal{BP}}
\newcommand{\Dir}{\textnormal{Dir}}
\newcommand{\Expo}{\textnormal{Expo}}
\newcommand{\Gam}{\textnormal{Gamma}}
\newcommand{\HypGeo}{\textnormal{HypGeo}}
\newcommand{\Mult}{\textnormal{Mult}}
\newcommand{\NegMult}{\textnormal{NegMult}}
\newcommand{\Poi}{\textnormal{Poi}}
\newcommand{\Pois}{\textnormal{Pois}}
\newcommand{\Unif}{\textnormal{Unif}}
\newcommand{\Gen}{\mathcal{G}} % the trajectory generator object

%-------------------------------------------------------------------------------------------------------------------------------------------------------------------------------------------------------------------
% Derivative symbols
%-------------------------------------------------------------------------------------------------------------------------------------------------------------------------------------------------------------------
\newcommand{\grad}{\nabla} % gradient
\newcommand{\Hess}{\nabla^2} % Hessian
\newcommand{\lapl}{\triangle} % Laplace operator / Laplacian
\newcommand{\deriv}[2]{\frac{d #1}{d #2}} % derivative
\newcommand{\pderiv}[2]{\frac{\partial #1}{\partial #2}} % partial derivative

%-------------------------------------------------------------------------------------------------------------------------------------------------------------------------------------------------------------------
% Probability and statistics macros
%-------------------------------------------------------------------------------------------------------------------------------------------------------------------------------------------------------------------
\newcommand{\eqdist}{\stackrel{d}{=}}
\newcommand{\todist}{\stackrel{d}{\to}}
\newcommand{\toprob}{\stackrel{p}{\to}}
\def\KL#1#2{\textnormal{KL}({#1}\Vert{#2})}
\def\independenT#1#2{\mathrel{\rlap{$#1#2$}\mkern4mu{#1#2}}}
%\def\indep{\perp\!\!\!\perp} % conditional independence
\newcommand{\iid}{\textrm{i.i.d.}\xspace}
\newcommand{\distiid}{\overset{\textrm{\tiny\iid}}{\dist}}
\newcommand{\distind}{\overset{\textrm{\tiny\textrm{indep}}}{\dist}}
%-------------------------------------------------------------------------------------------------------------------------------------------------------------------------------------------------------------------
% Optimization macros
%-------------------------------------------------------------------------------------------------------------------------------------------------------------------------------------------------------------------
\newcommand{\subdiff}{\partial} % subdifferential
\providecommand{\esssup}{\mathop\mathrm{ess\,sup}}
\providecommand{\argmax}{\mathop\mathrm{arg max}} % Defining math symbols
\providecommand{\argmin}{\mathop\mathrm{arg min}}
\providecommand{\arccos}{\mathop\mathrm{arccos}}
\providecommand{\dom}{\mathop\mathrm{dom}}
\providecommand{\diag}{\mathop\mathrm{diag}}
\providecommand{\tr}{\mathop\mathrm{tr}}
\providecommand{\abs}{\mathop\mathrm{abs}}
\providecommand{\card}{\mathop\mathrm{card}}
\providecommand{\sign}{\mathop\mathrm{sign}}
\providecommand{\conv}{\mathop\mathrm{conv}} % Convex hull
\def\rank#1{\mathrm{rank}({#1})}
\def\supp#1{\mathrm{supp}({#1})}

\providecommand{\minimize}{\mathop\mathrm{minimize}}
\providecommand{\maximize}{\mathop\mathrm{maximize}}
\providecommand{\subjectto}{\mathop\mathrm{subject\;to}}

%\renewcommand\eqref[1]{Eq.~(\ref{#1})}

\def\openright#1#2{\left[{#1}, {#2}\right)}
