\chapter{Discovering Hydrothermalism from Afar}
\label{app:perception}

\section{Method for Methane Measurement from Niskin Bottles}
\label{app:perception:methane}
A Los Gatos Research (LGR) Dissolved Gas Extraction Unit (DGEU) and Greenhouse Gas Analyzer (GGA) were used to process water collected by Niskin bottle samples during the transect, and report methane concentration estimates to be compared to the \emph{in situ} observation of normalized methane by SAGE mounted on the rosette. Measurements of methane made by the GGA are reported as the stabilized parts per million (ppm) reading provided by the instrument after consuming 3-5 L of seawater from each Niskin bottle, and are converted to nanomolar (nM) values by first computing the partial pressure of methane, and then computing molarity by estimating the solubility constant of methane using coincident measurements of salinity and temperature of the seawater at time of bottle sample collection as measured by the rosette CTD. The conversion from partial pressure to molarity is done using the \verb|gasex| Python library, publicly hosted at \url{https://github.com/boom-lab/gasex-python}. 

To transform GGA measurements in ppm to partial pressure, the DGEU cell pressure is used, such that $\text{ppm} \times \text{cell pressure} = \text{partial pressure}$. Additionally, gas extraction inefficiency is taken into consideration at this step; the DGEU does not perfectly extract gas across the membrane during sampling. Extraction efficiency is used to scale the GGA measurement of methane prior to computing the partial pressure estimate by 

\begin{equation}
    \label{eq:extraction}
    \left[\frac{x_{obs} - x_{ref}}{\lambda_{eff}} + x_{ref}\right]\frac{p_{cell}}{1000} = x_{pp}
\end{equation}

\noindent where $x_{obs}$ is the ppm measurement made by the GGA, $x_{ref}$ is a methane reference value (the atmospheric concentration of methane, typically between 1.86-1.99 ppm), $\lambda_{eff}$ is the extraction efficiency, $p_{cell}$ is the cell pressure in millibar, and $x_{pp}$ is the estimated partial pressure value, in $\mu$atm. 

The extraction efficiency used in this manuscript was estimated by laboratory calibrations to be between 2.3-3.3\%, consistent across different water temperatures and different test tank concentrations. In the laboratory calibration procedure, methane was bubbled in a temperature-controlled tank which was stirred before two discrete samples were taken using 60 mL syringes filled with 40 mL of water, and 20 mL of pure nitrogen gas. A DGEU, connected to the GGA, was then used to take water from the target tank, and ppm measurements by the GGA were recorded when measurements stabilized; this was done with two different DGEUs, which we label A and B. To estimate ``ground truth'' partial pressure of methane in the tank, the syringe samples were shaken for 2 minutes to extract the dissolved gas content, and the water drained. The samples were then processed within 24 hours on a gas chromatography instrument (Shimadzu GC-14B), run alongside a set of standards processed every 5 minutes. The measurements from the processed syringes (DGEU influent) were used as $x_{pp}$ in Eq.~\ref{eq:extraction}, the GGA observations as $x_{obs}$, the value 1.99 ppm used as $x_{ref}$, and 495 mbar as $p_{cell}$. The relevant data from these calibrations is available in Tab.~\ref{tab:extraction}. DGEU A was the instrument used in the transect field mission as presented in this manuscript.

\begin{table}[h!]
    \centering
    \begin{tabular}{c|c|c|c|c}
        DGEU & Temperature (C) & Influent ($\mu$atm) & GGA Methane (ppm) & Efficiency  \\
        \hline
        \hline
        & & & & \\
        A & 4.7 & 299.13 & 21.82 & 3.29\% \\
        A & 4.7 & 512.03 & 6.44 & 0.4\% \\
        A & 4.7 & 588.25 & 41.04 & 3.29\% \\
        B & 4.7 & 299.13 & 17.77 & 2.62\% \\
        B & 4.7 & 512.03 & 25.68 & 2.29\% \\
        B & 4.7 & 588.25 & 30.09 & 2.37\% \\
        A & 9.9 & 267.14 & 17.02 & 2.80\% \\
        A & 9.9 & 403.45 & 27.84 & 3.18\%  \\
        A & 9.9 & 856.89 & 55.77 & 3.11\% \\
        B & 9.9 & 267.14 & 12.72 & 2.00\% \\
        B & 9.9 & 403.45 & 18.63 & 2.05\% \\
        B & 9.9 & 856.89 & 36.99 & 2.02\% \\
        A & 14.8 & 18.64 & 2.78 & 2.22\% \\
        A & 14.8 & 1549.18 & 101.26 & 3.17\%  \\
        A & 14.8 & 1640.81 & 100.41 & 2.97\% \\
        B & 14.8 & 18.64 & 2.63 & 1.80\% \\
        B & 14.8 & 1549.18 & 78.93 & 2.46\% \\
        B & 14.8 & 1640.81 & 68.43 & 2.01\% \\
        & & & & \\
        
    \end{tabular}
    \caption{Results of DGEU extraction efficiency calibration experiments.}
    \label{tab:extraction}
\end{table}



\section{Leg 2 Niskin Bottle Sample Schedule and Measurements}
\label{app:perception:niskin}
This manuscript presents methane and ammonium measurements collected by Niskin bottles during Leg 2 of the rosette trajectory. Table~\ref{tab:niskin_sched} provides the schedule of Niskin bottle firing performed during Leg 2, and Table~\ref{tab:niskin_sched_results} provides all data associated with those bottles collected and presented in Chapter 3 of this thesis. The range of methane nM values is provided by converting GGA methane ppm measurements as described in Sec.~\ref{app:perception:methane} for the conservative range of valid DGEU extraction efficiency values.
\begin{table}[h!]
    \centering
    \begin{tabular}{c|c|c|c}
        Bottle & Time & Location & Depth (m) \\
        \hline
        \hline
        &&&\\
        1 & 2021-11-30 09:10:03 & 27.3951N 111.3649W & 1648.62 \\
        3 & 2021-11-30 09:30:03 & 27.3956N 111.3665W & 1625.67 \\
        5 & 2021-11-30 09:47:01 & 27.3967N 111.3696W & 1639.25 \\
        7 & 2021-11-30 09:47:05 & 27.3967N 111.3696W & 1639.05 \\
        9 & 2021-11-30 10:07:00 & 27.3985N 111.3740W & 1598.32 \\
        11 & 2021-11-30 10:17:02 & 27.2994N 111.3765W & 1580.5 \\
        13 & 2021-11-30 10:27:01 & 27.4005N 111.3791W & 1568.27 \\
        15 & 2021-11-30 10:27:04 & 27.4005N 111.3791W & 1568 \\
        17 & 2021-11-30 10:37:20 & 27.4016N 111.2818W & 1558.64 \\
        19 & 2021-11-30 10:46:59 & 27.4027N 111.3845W & 1553.92 \\
        21 & 2021-11-30 11:07:05 & 27.4051N 111.3900W & 1547 \\
        23 & 2021-11-30 11:33:00 & 27.4082N 111.3971W & 1545.4 \\
        &&&
    \end{tabular}
    \caption{Schedule of bottle samples during Leg 2 of rosette transect.}
    \label{tab:niskin_sched}
\end{table}

\begin{table}[h!]
    \centering
    \begin{tabular}{c|c|c|c|c|c}
        Bottle & CH$_4$ (ppm) & CH$_4$ (nM) & NH$_4^+$ (nM) & Temp. (C) & Salinity (PSU) \\
        \hline
        \hline
        &&&&&\\
        1 & -- & -- & 0.00 & 2.8334 & 34.6104 \\
        3 & 9.29 & 207-296 & 46.35 & 2.8578 & 35.6095 \\ 
        5 & 21.6 & 547-785 & -- & 2.8458 & 34.6107 \\
        7 & -- & -- & 174.48 & 2.8461 & 34.6108 \\
        9 & 22.54 & 573-821 & 165.99 & 2.8659 & 34.6101 \\
        11 & 29.82 & 774-1110 & 225.87 & 2.8719 & 34.6096 \\
        13 & 44.36 & 1176-1686 & -- & 2.8734 & 34.6099 \\
        15 & -- & -- & 384.28 & 2.8733 & 34.6098 \\
        17 & 89.45 & 2421-3473 & 780.53 & 2.8849 & 34.6105 \\
        19 & 114.27 & 3105-4454 & 997.45 & 2.8968 & 34.6111 \\
        21 & 27.29 & 704-1009 & 227.54 & 2.8835 & 34.6087 \\
        23 & 11.5 & 268-384 & 89.29 & 2.8964 & 34.6075 \\
        &&&&&
    \end{tabular}
    \caption[Geochemical measurements collected during leg 2 of rosette transect]{Geochemical measurements associated with the schedule of bottle samples during Leg 2 of rosette trajectory. Note that methane expressed in nM is computed using coincident temperature and salinity measurements during the transect as measured by rosette CTD, and extraction inefficiency of the DGEU is compensated for as described in Sec.~\ref{app:perception:methane}.}
    \label{tab:niskin_sched_results}
\end{table}


\section{Normalized Pythia Calibration}
\label{app:perception:norm}
The Pythia instrument provides a significantly nonlinear output reference value when measuring methane. We correct for this nonlinearity using a reference curve computed in the laboratory before normalizing the measurements as reported in this manuscript. The reference curve was created using a temperature-fixed (\SI{3}{\celsius}) tank and closed equilibriation chamber, in which methane standards were bubbled until fully equilibriated before being measured by the instrument. Stable measurements by Pythia (which has a response time of approximately 35 minutes) were then recorded at different chamber concentrations. The calibration curve that results is a piece-wise linear function, shown in Fig.~\ref{fig:nopp_curve}.

\begin{figure}[h!]
    \centering
    \includegraphics[width=0.6\textwidth]{figures/nopp_fundamental_calib_inverted.png}
    \caption{Fitted calibration curve for measurements of methane observed by Pythia.}
    \label{fig:nopp_curve}
\end{figure}

Compensation of Pythia's time response was also performed on post-calibrated data using the methodology described in \cite{miloshevich2004development} with a smoothing window of 5 minutes, and subsampling at a quarter of the time delay window. This methodology is sensitive to noise in the signal, which motivates the extreme sub-sampling that is performed. Fig.~\ref{fig:fund_corrected} shows the effect of smoothing, time-correction, and conversion on the direct signal recorded by Pythia before normalization. 

\begin{figure}[h!]
    \centering
    \includegraphics[width=1\textwidth]{figures/pythia_calibration.png}
    \caption[Pythia calibrated field data]{Calibration curve, smoothing, and time correction applied to Pythia observations during the transect, before reported normalization in the manuscript.}
    \label{fig:fund_corrected}
\end{figure}


\section{Depth-Correction}
\label{app:perception:depth}
Temperature, salinity, and oxygen are expected to be weakly stratified in the deep ocean. To remove these effects from data collected by AUV Sentry and the rosette, we fit a line to the average observations collected within binned 20 m intervals of observed depth for each platform separately. Separately computing the correction for each instrument additionally controls for small discrepancies in calibration between the platforms. Fig.~\ref{fig:linear_fits} compares these lines with the observations collected.

\begin{figure}[h!]
    \centering
    \includegraphics[width=1\columnwidth]{figures/depth_correction_plots.png}
    \caption[Functions used for depth normalization]{Linear functions are fit to data collected for oxygen, temperature, and salinity instruments on each platform separately. A residual value is then computed for each observation.}
    \label{fig:linear_fits}
\end{figure}

\section{Description of Plume Model for Transect Design}
\label{app:perception:model}
We adapted an idealized buoyant bent-plume model proposed by \cite{tohidi2016highly} for atmospheric bent plumes in a weakly stratified fluid in order to inform at what heights to deploy AUV Sentry and the rosette during the transect. We rewrite the system of equations provided in \cite{tohidi2016highly} as follows:

\begin{align}
    E &= \alpha\left|\frac{M}{Q} - u\cos(\theta)\right| + \beta\left|u\sin(\theta)\right| \\
    \frac{dQ}{ds} &= QE\sqrt{\frac{2(1 + \lambda^2)}{M\lambda}} \\
    \frac{dM}{ds} &= u\cos(\theta)\frac{dQ}{ds} + \frac{FQ}{M}\sin(\theta)\\
    \frac{d\theta}{ds} &= \left(\frac{FQ}{M}\cos(\theta) - u\sin(\theta)\frac{dQ}{ds}\right)\frac{1}{M}
\end{align}
\begin{align}
    \frac{dF}{ds} &= -QN^2\sin(\theta)\\
    \frac{dX}{ds} &= \cos(\theta)\\
    \frac{dZ}{ds} &= \sin(\theta)
\end{align}

\noindent where $E$ is a mixing entrainment coefficient which considers both vertical and horizontal mixing and is weighted by parameters $\alpha$ and $\beta$, $u$ is the crossflow velocity which can be a function of depth and time, $\lambda$ is a parameter which modifies the ellipse which describes the plume envelope, $Q$ is specific volume flux, $M$ is specific momentum flux, $F$ is specific buoyancy flux, $\theta$ is plume centerline trajectory angle, $s$ is the plume centerline trajectory, $X$ is distance along a coordinate axis aligned with the plume centerline, $Z$ is height with respect to plume source along a vertical axis, and $N^2$ is the Brunt-V\"ais\"al\"a frequency, computed with respect to the density gradient at the reference depths of the source and plume height.

The system of equations essentially yields a ``snapshot'' of a plume envelope at some moment in time. For time-varying crossflows, multiple snapshots can be computed for different moments in time (different crossflow orientations and magnitudes) and chained together in a common coordinate reference system in order to track a plume trajectory. For the purposes of determining which heights to deploy AUV Sentry and the rosette for the transect, we compute a prototypical envelope and use the estimated bent nonbuoyant plume height to set the transect depths/altitudes.

The initial conditions for solving this system of ordinary differential equations are set via estimates of vent characteristics including exit velocity, temperature, salinity, and area. Specifically:

\begin{align}
    Q_o &= \lambda V_v \frac{A_v}{\pi} \\
    M_o &= Q_o V_v \\
    F_o &= -g10^{-4}(T_v - T_z)Q_o \\
    \theta_o &= \frac{\pi}{2}
\end{align}

\noindent where $V_v$ is exit velocity at the vent orifice, $A_v$ is the vent orifice area, $T_v$ is the temperature at the orifice area, and $T_z$ is the expected temperature of ambient seawater at the estimated vent depth. Note that initial buoyancy flux is primarily driven by temperature changes, as we anticipate this to be the major driver of density gradients at our measurement scale. Expected salinity gradients could be similarly considered.

Estimated vent characteristics and crossflow were selected based on empirical observations of the deep sea vents located along the northern Guaymas Basin ridge and observations of current magnitude collected by a current tiltmeter deployed by ROV Jason during several days of the research cruise. Table~\ref{tab:params} lists the settings for planning the transect selected for these characteristics. Background salinity and temperature profiles were computed according to standard Pacific Ocean temperature and salinity functions as described in \cite{speer1989model}; additionally the equation of state for computing density profile from salinity and temperature measurements was used also as defined in \cite{speer1989model}. The prototypical plume is computed with a source located at 1850 m depth.

\begin{table}[h!]
    \centering
    \begin{tabular}{c|c|l}
        Parameter & Assignment & Description \\
        \hline
        $\lambda$ & 1.0 & Ratio of elliptical axes of the plume envelope \\
        $V_v$ & \SI{0.58}{\meter\per\second} & Exit velocity of fluids at vent orifice \\
        $A_v$ & \SI{0.82}{\meter\squared} & Area of vent orifice \\
        $T_v$ & \SI{340}{\celsius} & Temperature of fluids at vent orifice \\
        $\alpha$ & 0.15 & Longitudinal shear-driven mixing coefficient \\
        $\beta$ & 0.19 & Transverse shear-driven mixing coefficient \\
        $u$ & \SI{0.1}{\meter\per\second} & Magnitude of crossflow \\
    \end{tabular}
    \caption{Parameter, vent characteristics, and ambient crossflow setting used for transect design.}
    \label{tab:params}
\end{table}

The prototypical plume envelope computed in this manner estimates a nonbuoyant plume depth between 1570-1750 m (Fig.~\ref{fig:plume_envelopes}). AUV Sentry is altitude limited in order to keep a fix on the ocean floor for navigation; it is set to its maximum altitude of \SI{120}{\meter} in order to intersect with the bottom of the estimated nonbuoyant layer; this corresponds to a depth of approximately 1700 m throughout the basin. The rosette can be arbitrarily fixed to a height, but so as not to interfere with AUV Sentry operations and to sample a different point in the estimated nonbuoyant layer, a depth of 1650-1600 m was targeted.

\begin{figure}[h!]
    \centering
    \includegraphics[width=1\columnwidth]{figures/plume_envelopes.png}
    \caption[Plume model for transect design]{A prototypical plume estimate according to the modified buoyant plume model in crossflow. The same envelope is plotted with respect to absolute depth (with a source located at 1850 m) on the left, and illustratively in the context of the hydrothermal ridge on the right.}
    \label{fig:plume_envelopes}
\end{figure}

\section{Code and Data Availability}
Ship-based and towed rosette data can be found in the Rolling Deck to Repository (R2R) at \url{https://doi.org/10.7284/909325}. AUV Sentry, SAGE, and Pythia data presented in this study are available on the Woods Hole Open Access Server (WHOAS) at \url{https://hdl.handle.net/1912/29403} with DOI 10.26025/1912/29403. Software and data analysis tools are available on Github at \url{https://doi.org/10.5281/zenodo.6789105}.

%%%%%%
% Appendix B
%%%%%%
\chapter{\PHORTEX Performance}
\label{app:phortex}

\section{Convergence of Trajectory Optimizer}
The trajectory optimization scheme presented in \cref{sec:to} uses a gradient-based optimizer with trust-bounded constraints in order to set the parameters of a set of lawnmower trajectories. In general, this is a difficult optimization problem, as analytical gradients that map the defining parameters of a lawnmower trajectory (orientation, size, and location) are not available, and must be numerically approximated. Moreover, by defining trust constraints, jumping between minima could be made more difficult, as the approximated gradients may lead through unsafe (un-trusted) space. In practice, this meant that initializing each lawnmower in a chain to be near a ``good trajectory'' was important. In the case of charting hydrothermal plumes, a ``good'' trajectory would be one approximately aligned with the estimated crossflow. As \PHUMES provides complete access to this information, it is generally easy to seed the trajectories to be near a performant minima. This leads to relatively fast convergence of the optimizer for each element in the chain. In \cref{fig:phortex_chain} there is an example a snapshot along the path to convergence for each link in a chain visualized, and the convergence plot. As is evidenced in the plot, a long travel time between chains is accumulated; this is likely because ``flipping'' a trajectory to shorten this distance would be a prohibitively difficult step in the nonlinear, constrained space that the optimizer operates.

\begin{figure}[h!]
    \centering
    \includegraphics[width=0.7\columnwidth]{figures/phortex_iterations.png}
    \caption[\PHORTEX optimization performance]{Example snapshots of each link in a \PHORTEX trajectory from a \cref{chap:phortex} simulation trial, and their corresponding convergence plots. While trajectories show good agreement with the current function and size of a plume expression as modeled by \PHUMES, a long travel time between chains is accumulated; this is likely because ``flipping'' a trajectory to shorten this distance would be a prohibitively difficult step in the nonlinear, constrained space that the optimizer operates.}
    \label{fig:phortex_chain}
\end{figure}

\section{MCMC Chain Characteristics of \PHUMES}
For computational reasons, it is typically not possible to run the chains for thousands of samples while in the field, and the chains must be arbitrarily cut short in order to generate a plan. For this reason, it is interesting to look at the chaining properties. For a simulation trial, as described in \cref{chap:phortex}, we show the chains and probability densities for each of vent area, vent fluid velocity, and entrainment coefficients in \cref{fig:phumes_sim_chain}. As evidenced by the multimodal area plot, we observe that the relationship between the parameters is complicated; in general, an inverse problem like the one we have posed for \PHUMES is a challenge to infer. What is promising however is that probability densities are being concentrated in areas that make sense (as observed by a mode at 0.8 for area), and that the \emph{combinations} of learned parameters are sensible --- a large area should be coupled with a slow or medium velocity; a high velocity should be paired with a small area --- based on the relationships between parameters and how they manifest in the ultimate shape of the plumes in the 3D Cartesian space.

\begin{figure}[h!]
    \centering
    \includegraphics[width=1\columnwidth]{figures/phumes_trial_chain.png}
    \caption[\PHUMES simulation chain]{An example chain from a simulation trial as described in \cref{chap:phortex}. The last 150 samples in each chain is used to compute the distributions in blue; all samples are shown in gray in the top plots. Red lines in the top plots indicate the generating environment value for the simulation. The chain efficiency is approximately 37.5\%.}
    \label{fig:phumes_sim_chain}
\end{figure}

We can also look at how convergence may change for longer chains. In an exemplar trial (\cref{fig:phumes_long_chain}), a chain is run for 650 samples, and the last 500 samples are used to compute densities. The acceptance rate stays approximately the same, at about 34\%. However, the distributions have generally improved with respect to placing more density at the parameters that defined the true underlying environment. As evidenced in the area and velocity sampling plots, there is still some exploration of the state space being performed by the chains; this suggests that true convergence of these chains may either take a long time, or the information content of the data that is available to explore the state space leads to ambiguous ``wells'' which the chains would ``bounce'' between for long durations. This suggests that other Monte Carlo techniques, such as Hamiltonian Monte Carlo~\autocite{duane1987hybrid}, may be of interest in adaptation of this work, to accelerate chain exploration.

\begin{figure}[h!]
    \centering
    \includegraphics[width=1\columnwidth]{figures/phumes_long_chain.png}
    \caption[\PHUMES long chain]{An example chain from a simulation trial as described in \cref{chap:phortex}, but run for 650 samples; the last 500 are used to compute the distributions in blue with all samples shown in gray in the top plots. Red lines in the top plots indicate the generating environment value for the simulation. The chain efficiency is approximately 34\%.}
    \label{fig:phumes_long_chain}
\end{figure}

\section{Code Availability}
\PHUMES and \PHORTEX is available through the Expeditionary Robotics organization on GitHub, \url{https://github.com/expeditionary-robotics}.


%%%%%%%%%%%%%%%%%%%%%
% Field Work Appendix
%%%%%%%%%%%%%%%%%%%%%
\chapter{Guaymas Basin Field Results}

\section{Instrument Data}
In the main body of the text, the binary detections are reported for each of four AUV \Sentry dives. Here, the continuous data is plotted on overhead polar charts for reference.

\begin{sidewaysfigure}
    \centering
    \includegraphics[width=1\columnwidth]{figures/app_sentry607_data.png}
    \caption[AUV \Sentry dive 607 data.]{\textbf{AUV \Sentry dive 607 data.} The binary detections are plotted with methane value from the Pythia methane instrument (time-corrected and converted data to nM), turbidity, ORP, detrended oxygen, and detrended temperature.}
    \label{fig:app:field:sentry607}
\end{sidewaysfigure}


\begin{sidewaysfigure}
    \centering
    \includegraphics[width=1\columnwidth]{figures/app_sentry608_data.png}
    \caption[AUV \Sentry dive 608 data.]{\textbf{AUV \Sentry dive 608 data.} The binary detections are plotted with methane value from the Pythia methane instrument (time-corrected and converted data to nM), turbidity, ORP, detrended oxygen, and detrended temperature.}
    \label{fig:app:field:sentry608}
\end{sidewaysfigure}


\begin{sidewaysfigure}
    \centering
    \includegraphics[width=1\columnwidth]{figures/app_sentry610_data.png}
    \caption[AUV \Sentry dive 610 data.]{\textbf{AUV \Sentry dive 610 data.} The binary detections are plotted with methane value from the Pythia methane instrument (time-corrected and converted data to nM), turbidity, ORP, detrended oxygen, and detrended temperature.}
    \label{fig:app:field:sentry610}
\end{sidewaysfigure}

\begin{sidewaysfigure}
    \centering
    \includegraphics[width=1\columnwidth]{figures/app_sentry611_data.png}
    \caption[AUV \Sentry dive 611 data.]{\textbf{AUV \Sentry dive 611 data.} The binary detections are plotted with methane value from the SAGE methane instrument (time-corrected and converted data to nM), turbidity, ORP, detrended oxygen, and detrended temperature.}
    \label{fig:app:field:sentry611}
\end{sidewaysfigure}

\newpage
\section{Acoustically Broadcast Science Data}
During the cruise, a custom live-plotter for acoustically transmitted oceanographic sensing equipment from AUV \Sentry was developed. The open source code for this plotter can be accessed at \url{https://github.com/expeditionary-robotics/auv_listener}. While acoustically transmitted data is severely sub-sampled, the plots can be useful for showing general trends. Here, an example of the data that was transmitted by \Sentry during dive 607 and the absolute data from that dive are shown in \cref{fig:app:field:plotter}.

\begin{figure}[h!]
    \centering
    \includegraphics[width=1\columnwidth]{figures/acomms.png}
    \caption[Acoustic versus logged OBS messages during AUV \Sentry dive 607]{\textbf{Acoustic versus logged OBS messages during AUV \Sentry dive 607.} Red points mark the acoustic messages that were transmitted over the acoustic \Sentry network, plotted, and logged to a shipside computer. Most of the gross structure of the OBS data stream was captured, potentially indicating that the acoustic messages, although severely subsampled, may be sufficient for real-time analysis of vehicle performance.}
    \label{fig:app:field:plotter}
\end{figure}

\section{Code and Data Availability}
Rosette data can be found in the Rolling Deck to Repository (R2R) at \url{https://doi.org/10.7284/909325}. Data analysis tools used for processing \Sentry data are available on Github at \url{https://doi.org/10.5281/zenodo.6789105}. \Sentry, ROV \emph{JASON}, and tiltmeter data will be made available at a future date. Please contact the author with questions.

\newpage
\section{Autonomy Field Report}
During RR2107, every science team generated a field report of activities as a reference for post-cruise analysis. While much of the report by the autonomy team on the RR2107 has been included throughout the main thesis body, the report as it was generated on the ship, is included here for completeness and is part of the RR2107 overall cruise report. 

\includepdf[pages=-,width=1.2\textwidth,pagecommand={}]{appendix/prestonflaspohler_report.pdf}