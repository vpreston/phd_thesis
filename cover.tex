% -*-latex-*-
% 
% For questions, comments, concerns or complaints:
% thesis@mit.edu
% 
%
% $Log: cover.tex,v $
% Revision 1.9  2019/08/06 14:18:15  cmalin
% Replaced sample content with non-specific text.
%
% Revision 1.8  2008/05/13 15:02:15  jdreed
% Degree month is June, not May.  Added note about prevdegrees.
% Arthur Smith's title updated
%
% Revision 1.7  2001/02/08 18:53:16  boojum
% changed some \newpages to \cleardoublepages
%
% Revision 1.6  1999/10/21 14:49:31  boojum
% changed comment referring to documentstyle
%
% Revision 1.5  1999/10/21 14:39:04  boojum
% *** empty log message ***
%
% Revision 1.4  1997/04/18  17:54:10  othomas
% added page numbers on abstract and cover, and made 1 abstract
% page the default rather than 2.  (anne hunter tells me this
% is the new institute standard.)
%
% Revision 1.4  1997/04/18  17:54:10  othomas
% added page numbers on abstract and cover, and made 1 abstract
% page the default rather than 2.  (anne hunter tells me this
% is the new institute standard.)
%
% Revision 1.3  93/05/17  17:06:29  starflt
% Added acknowledgements section (suggested by tompalka)
% 
% Revision 1.2  92/04/22  13:13:13  epeisach
% Fixes for 1991 course 6 requirements
% Phrase "and to grant others the right to do so" has been added to 
% permission clause
% Second copy of abstract is not counted as separate pages so numbering works
% out
% 
% Revision 1.1  92/04/22  13:08:20  epeisach

% NOTE:
% These templates make an effort to conform to the MIT Thesis specifications,
% however the specifications can change. We recommend that you verify the
% layout of your title page with your thesis advisor and/or the MIT 
% Libraries before printing your final copy.
\title{Spatiotemporal Forecasting for Robotic Adaptive Sampling}

\author{Victoria Lynn Preston}
% If you wish to list your previous degrees on the cover page, use the 
% previous degrees command:
%       \prevdegrees{A.A., Harvard University (1985)}
% You can use the \\ command to list multiple previous degrees
%       \prevdegrees{B.S., University of California (1978) \\
%                    S.M., Massachusetts Institute of Technology (1981)}
\prevdegrees{B.S., Olin College of Engineering (2016) \\
             M.S., Massachusetts Institute of Technology (2019)}
\department{Department of Aeronautics and Astronautics, MIT}
\whoidepartment{Applied Ocean Science and Engineering, WHOI}

% If the thesis is for two degrees simultaneously, list them both
% separated by \and like this:
% \degree{Doctor of Philosophy \and Master of Science}
\degree{Doctor of Philosophy}

% As of the 2007-08 academic year, valid degree months are September, 
% February, or June.  The default is June.
\degreemonth{February}
\degreeyear{2023}
\thesisdate{December 15, 2022}

%% By default, the thesis will be copyrighted to MIT.  If you need to copyright
%% the thesis to yourself, just specify the `vi' documentclass option.  If for
%% some reason you want to exactly specify the copyright notice text, you can
%% use the \copyrightnoticetext command.  
%\copyrightnoticetext{\copyright IBM, 1990.  Do not open till Xmas.}

% If there is more than one supervisor, use the \supervisor command
% once for each.
\supervisor{Nicholas Roy}{Professor of Aeronautics and Astronautics, MIT}
\supervisor{Anna Michel}{Associate Scientist with Tenure, Applied Ocean Physics and Engineering, WHOI}

% This is the department committee chairman, not the thesis committee
% chairman.  You should replace this with your Department's Committee
% Chairman.
\chairman{Jonathan How}{Professor of Aeronautics and Astronautics, MIT}{Chair, Graduate Committee}
\whoichairman{David Ralston}{Associate Scientist with Tenure, Applied Ocean Physics and Engineering, WHOI}{Chair, Joint Committee on Applied Ocean Science and Engineering}

% Make the titlepage based on the above information.  If you need
% something special and can't use the standard form, you can specify
% the exact text of the titlepage yourself.  Put it in a titlepage
% environment and leave blank lines where you want vertical space.
% The spaces will be adjusted to fill the entire page.  The dotted
% lines for the signatures are made with the \signature command.
\maketitle

% The abstractpage environment sets up everything on the page except
% the text itself.  The title and other header material are put at the
% top of the page, and the supervisors are listed at the bottom.  A
% new page is begun both before and after.  Of course, an abstract may
% be more than one page itself.  If you need more control over the
% format of the page, you can use the abstract environment, which puts
% the word "Abstract" at the beginning and single spaces its text.

%% You can either \input (*not* \include) your abstract file, or you can put
%% the text of the abstract directly between the \begin{abstractpage} and
%% \end{abstractpage} commands.

% First copy: start a new page, and save the page number.
\cleardoublepage
% Uncomment the next line if you do NOT want a page number on your
% abstract and acknowledgments pages.
% \pagestyle{empty}
\setcounter{savepage}{\thepage}
\begin{abstractpage}
\begin{center}
    {\large \@title} \\
    \emph{\footnotesize by} \\
    \@author \\
    \end{center}
    
    \vspace{-1.5em}
    
    \begin{center}
    \begin{singlespace}
    {\parindent0pt
    \small
    Submitted to the Department of Aeronautics and Astronautics and the Joint Program in Applied Ocean Science \& Engineering on \@date ~in partial fulfillment of the requirements for the degree of Doctor of Philosophy}
    \end{singlespace}
    \end{center}
    
    \begin{singlespace}
    {\parindent0pt 
        {\large \textsc{Abstract}} \\ %less than 200 words for WHOI (350 for MIT)
    
        % Context:
        Transient, dynamic phenomena are of interest in many of the disciplines of observational science. \emph{Expeditionary science} encapsulates the observational sciences that require \emph{in situ} sample collection of environmental phenomena.
        % Importance:
        Taking physical measurements of the natural world and building accurate models from those observations is crucial to understanding a phenomenon and the spatiotemporal processes that drive it. Autonomous robots are especially well-suited for gathering those measurements efficiently.
        % Challenge:
        However, to collect useful observations of unknown, partially-observed spatiotemporal distributions for scientific inquiry requires accurately perceiving a phenomenon of interest, predicting how it will evolve in time, and planning effective sampling trajectories to put the robot in the right place at the right time, potentially under severe operational constraints.
        % Intuition/Thesis Statement:
        % In this thesis, the core challenges of \emph{extreme partial observability} and \emph{limited adaptive intelligence} are overcome by introducing methods embedded with domain scientific expertise to gain computational tractability with operating in the field.
        Key to addressing these challenges is uncovering the underlying physics which describe phenomenon evolution.
        This thesis establishes algorithmic contributions for computing tractable, probabilistic  solutions to inverse problems for long-horizon forward simulation of a target environment.
        To do so, idealized scientific summaries of a given target environment for a particular task are formulated and embedded as strong inductive priors into learning frameworks suitable for use in robotic informative path planning.
        % How Intuition was Applied:
        To ground the algorithmic discussion, the problem of charting deep-sea hydrothermal plumes with a robot restricted to executing non-adaptive behaviors is used as a scientific and technical context for development.
        As such, this thesis presents novel methods for detecting hydrothermal plumes from heterogeneous instruments, models for utilizing those observations to uncover and forward simulate plume dynamics, and a trajectory optimization strategy suitable for state-of-the-art expeditionary robots in deep-sea oceanographic research.
        % Results:
        Each chapter will discuss the technical and scientific implications of the proposed methods, drawing on results from a real field trial for mapping hydrothermal plumes with autonomous underwater vehicle (AUV) \emph{Sentry} in the Gulf of California at the Guaymas Basin in 2021.
        % Impact:
        The outcome of this thesis is an illustrative example of an autonomy system which extends the capabilities of expeditionary robots towards previously unattainable queries about dynamic phenomena.\\
    }
    
        \noindent Thesis Supervisor: Nicholas Roy \\
        \noindent Title: Bisplinghoff Professor of Aeronautics and Astronautics, MIT \\

        \noindent Thesis Supervisor: Anna Michel \\
        \noindent Title: Associate Scientist with Tenure, Applied Ocean Physics and Engineering, WHOI

    \end{singlespace}
    
    \newpage
    \null
    \thispagestyle{empty}
    \newpage

% To enable modern expeditionary robots to perform complex tasks under limited self-agency requires a tractable means of discovering the underlying physics of a target phenomenon and usefully forward simulating it over long-horizons for trajectory planning. 
% Solving inverse problems of natural environments is computationally taxing, and impractical for use in field applications.
% Moreover, discovering physics from scratch using data, let alone noisy, partial observations, is also computationally taxing.
% But the intuition here is that we need to be able to do something useful, so we embed science as an inductive bias into data-driven models which perform simplified dynamical forward simulations.
\end{abstractpage}

% Additional copy: start a new page, and reset the page number.  This way,
% the second copy of the abstract is not counted as separate pages.
% Uncomment the next 6 lines if you need two copies of the abstract
% page.
% \setcounter{page}{\thesavepage}
% \begin{abstractpage}
% \begin{center}
    {\large \@title} \\
    \emph{\footnotesize by} \\
    \@author \\
    \end{center}
    
    \vspace{-1.5em}
    
    \begin{center}
    \begin{singlespace}
    {\parindent0pt
    \small
    Submitted to the Department of Aeronautics and Astronautics and the Joint Program in Applied Ocean Science \& Engineering on \@date ~in partial fulfillment of the requirements for the degree of Doctor of Philosophy}
    \end{singlespace}
    \end{center}
    
    \begin{singlespace}
    {\parindent0pt 
        {\large \textsc{Abstract}} \\ %less than 200 words for WHOI (350 for MIT)
    
        % Context:
        Transient, dynamic phenomena are of interest in many of the disciplines of observational science. \emph{Expeditionary science} encapsulates the observational sciences that require \emph{in situ} sample collection of environmental phenomena.
        % Importance:
        Taking physical measurements of the natural world and building accurate models from those observations is crucial to understanding a phenomenon and the spatiotemporal processes that drive it. Autonomous robots are especially well-suited for gathering those measurements efficiently.
        % Challenge:
        However, to collect useful observations of unknown, partially-observed spatiotemporal distributions for scientific inquiry requires accurately perceiving a phenomenon of interest, predicting how it will evolve in time, and planning effective sampling trajectories to put the robot in the right place at the right time, potentially under severe operational constraints.
        % Intuition/Thesis Statement:
        % In this thesis, the core challenges of \emph{extreme partial observability} and \emph{limited adaptive intelligence} are overcome by introducing methods embedded with domain scientific expertise to gain computational tractability with operating in the field.
        Key to addressing these challenges is uncovering the underlying physics which describe phenomenon evolution.
        This thesis establishes algorithmic contributions for computing tractable, probabilistic  solutions to inverse problems for long-horizon forward simulation of a target environment.
        To do so, idealized scientific summaries of a given target environment for a particular task are formulated and embedded as strong inductive priors into learning frameworks suitable for use in robotic informative path planning.
        % How Intuition was Applied:
        To ground the algorithmic discussion, the problem of charting deep-sea hydrothermal plumes with a robot restricted to executing non-adaptive behaviors is used as a scientific and technical context for development.
        As such, this thesis presents novel methods for detecting hydrothermal plumes from heterogeneous instruments, models for utilizing those observations to uncover and forward simulate plume dynamics, and a trajectory optimization strategy suitable for state-of-the-art expeditionary robots in deep-sea oceanographic research.
        % Results:
        Each chapter will discuss the technical and scientific implications of the proposed methods, drawing on results from a real field trial for mapping hydrothermal plumes with autonomous underwater vehicle (AUV) \emph{Sentry} in the Gulf of California at the Guaymas Basin in 2021.
        % Impact:
        The outcome of this thesis is an illustrative example of an autonomy system which extends the capabilities of expeditionary robots towards previously unattainable queries about dynamic phenomena.\\
    }
    
        \noindent Thesis Supervisor: Nicholas Roy \\
        \noindent Title: Bisplinghoff Professor of Aeronautics and Astronautics, MIT \\

        \noindent Thesis Supervisor: Anna Michel \\
        \noindent Title: Associate Scientist with Tenure, Applied Ocean Physics and Engineering, WHOI

    \end{singlespace}
    
    \newpage
    \null
    \thispagestyle{empty}
    \newpage

% To enable modern expeditionary robots to perform complex tasks under limited self-agency requires a tractable means of discovering the underlying physics of a target phenomenon and usefully forward simulating it over long-horizons for trajectory planning. 
% Solving inverse problems of natural environments is computationally taxing, and impractical for use in field applications.
% Moreover, discovering physics from scratch using data, let alone noisy, partial observations, is also computationally taxing.
% But the intuition here is that we need to be able to do something useful, so we embed science as an inductive bias into data-driven models which perform simplified dynamical forward simulations.
% \end{abstractpage}

\cleardoublepage

\section*{Acknowledgments}

Thank you, everyone.

%%%%%%%%%%%%%%%%%%%%%%%%%%%%%%%%%%%%%%%%%%%%%%%%%%%%%%%%%%%%%%%%%%%%%%
% -*-latex-*-
