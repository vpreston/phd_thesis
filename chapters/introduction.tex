%%%%%%%
% This chapter is all about setting the vision for the work, and listing contributions 
%%%%%%%
\chapter{Introduction}
\label{chap:intro}

\begin{center}
    \begin{minipage}{0.7\textwidth}
      \begin{small}
       It is a wholesome and necessary thing for us to turn again to the earth and in the contemplation of her beauties to know the sense of wonder and humility.\\ \emph{Rachel Carson}
      \end{small}
    \end{minipage}
    \vspace{0.5cm}
\end{center}


% Our pale blue dot\footnote{a term coined by Carl Sagan reflecting on a photo Voyager I snapped of Earth on it's trek out of the Solar System} is home to broad classes of flora and fauna.
% That blue comes from the hue of the planet's liquid water oceans, which cover the majority of the planet's surface.

The environmental sciences are the multidisciplinary, academic studies which aim to understand the Earth and its processes.
\emph{In situ} observational studies, or \emph{expeditions}, serve as the foundation on which scientific discovery and model development are predicated in these fields.
With improvements in technology, expeditions have been conducted in the deepest trenches of the ocean to the uppermost atmosphere.
Improved reach, in addition to improved observational quality, density, and availability has made it increasingly clear how inextricably entwined Earth's regulatory processes are, and how crucial a role the ocean plays in these processes.
Covering 70\% of the Earth's surface, the ocean is the largest biosphere on the planet, home to a staggering number of unique microorganisms up to the largest creatures on Earth, ultimately encompassing over 90\% of the habitable volume on Earth\autocite{cario2019exploring,purkis2022remote}.
The plant life supported by the ocean and ocean-coast interfaces are estimated to absorb 50\% of all excess carbon dioxide emissions produced by anthropogenic sources, acting as a buffer to global heating\autocite{hori2019blue}.
Culturally, the ocean is also entwined with our sense of humanity and development of society---island and coastal habitats supported the earliest hominids\autocite{erlandson2006oceans}, traveling the ocean has shaped trade, conquest, and tradition\autocite{pearson2003indian,chaudhuri1985trade,firth2019understanding,nunn2003nature}, and the ocean inspires creativity, recreation, and curiosity.
Despite the centrality of the ocean to existence (and continued existence) as we know it on Earth, there is yet so much that has yet to be discovered.

In the last decade, significant effort has been put to finely mapping the seafloor\footnote{e.g., through Seabed 2030~\autocite{mayer2018nippon}, among other initiatives}, and as of mid-2022 nearly a quarter of the seafloor has been mapped bathymetrically in high resolution (compared to about 6\% in 2017)\footnote{As reported by Seabed 2030 at \url{https://seabed2030.org/mapping-progress}.}. 
High resolution maps allow us to better understand continental drift and crustal processes, the distribution of natural resources, and the ecosystems which ocean structures can support.
Unlike terrestrial environments, which can be largely observed remotely (either by aircraft, or more commonly now, satellite), the ocean, especially the deep ocean, cannot be globally observed due to the conductive properties of water and it's tendency to absorb many forms of light and radio energy.
Mapping the seafloor requires physically going to sea.
The mapping revolution of the ocean is enabled thanks in part\footnote{In other part, it has been enabled by an increased economic and political commitment to the ocean, it's strategic value, and the resources contained within it.} to improved acoustic technology and processing tools, which allow shipboard acoustic sounders to collect high resolution ``imagery'' while traversing on the surface ocean.

Bathymetry is one piece of the giant puzzle that is the ocean; another piece seeing contemporary scrutiny is geochemistry of the deep ocean.
Geochemistry, the study of Earth and other planetary geological systems through chemical principles, enables us to understand the processes which create and sustain the structures that bathymetric maps reveal, and tell us more about local ecosystems and their nutrient and energetic budgets. 
Geochemical measurements may range from studying the composition of rock samples from the seafloor, to \emph{in situ} observations of dissolved gases in a hydrothermal plume.
Modern interests in seafloor mining to remove materials from the deep ocean\autocite{thompson2018seabed}, and deep ocean carbon sequestration to inject the ocean with excess materials\autocite{teng2018long}, stand to directly impact the balance of the deep biogeosphere, with implications that have yet to be well understood (or agreed upon) by science\autocite{smith2020deep,seibel2001potential,fleeger2010response,sharma2015environmental,childs2020extraction,van2011tighten}.
Unlike acoustic surveys, which even though highly localized can be performed from a ship hundreds or thousands of meters above the seabed, geochemical surveys must truly be conducted \emph{in situ} with instruments physically sent to the deep ocean to collect continuous measurements, or physical bottle samples of water or rock specimens retrieved for \emph{ex situ} analysis.
There is an ongoing paradigm shift in ocean technology to better enable geochemical studies of the deep ocean: development of novel \emph{in situ} sensors, creation of more depth-capable instruments, and adoption of autonomous technologies for sample collection or expedition planning. 

Intelligent autonomous technologies are any systems which can automatically process and analyze a data product, and formulate evidence-based decisions which they may then act upon.
These systems may be \emph{embodied}, like robotic vehicles, or simply algorithmic.
A science party member at sea can be viewed as an intelligent autonomous agent for an expedition: the scientist will be collecting and processing data throughout the voyage, using that data to coordinate with other science party members to design activities, and managing the deployment of various sensors and platforms.
Good algorithmic or robotic contributions relieve the burden of data processing and decision-making on the science party to make their time, the expedition time, and the resources aboard a vessel more effective.
For marine geochemical surveys, one of the key challenges for the science party to grapple with is understanding \emph{spatiotemporal distributions}, which are nearly ubiquitous in water column studies.
A spatiotemporal distribution is a phenomenon that evolves in space and time; in the deep sea, these may take the form of hydrothermal plumes, hydrate dissolution, sediment transport, or water mass mixing.

\emph{Perceiving} a spatiotemporal distribution from sparse \emph{in situ} measurements; \emph{predicting} the evolution of the distribution into future expedition dates or sites; and \emph{planning} where and when to take samples of the phenomenon for further inspection, are key challenges for any autonomous agent.
This thesis presents several algorithmic and operational strategies for addressing these challenges for deep ocean geochemical surveys, with a particular contextual focus on charting the spatiotemporal structure of hydrothermal plumes.
The algorithmic contributions of this thesis are motivated by tools used in the discipline of robotics to enable \emph{planning under uncertainty}; a technique for decision-making with incomplete information about a target environment or task.
In planning under uncertainty architectures there are many design choices available to an engineer---how data is processed, how the data is used to construct a \emph{belief} about the environment or task, and how that belief is leveraged to pick actions to take.
Field and scientific contexts provide constraints and requirements that shape the feasible set of design choices that can be made, sometimes requiring creativity outside the pale in theoretical research.
A core tenant of the research presented here is a field roboticist has a special responsibility to ensure that the solutions that are engineered accomplish the science that they were designed for.
In line with this, scientific results which enhance understanding about deep sea hydrothermal plumes are found throughout this thesis, in balance with formulation and assessment of autonomous technologies.

In the rest of this chapter, an overview of deep hydrothermalism and geochemical field operations is provided, in addition to a summary of specific algorithmic challenges and the techniques employed in this thesis to overcome them.
A brief contribution statement concludes the chapter.

% This research spans practical to theoretical challenges and considerations, and was performed in close collaboration with scientists and engineers in fields including oceanography and geochemistry, computational statistics, robotics, and data science.
% Chief among the challenges addressed in this research is the problem of uncovering the underlying dynamics of a spatiotemporal distribution.


%%%%%%%%%%%%%%%%%%%%%%%%%%%%%%%%%%
% Hydrothermal Plume Charting
%%%%%%%%%%%%%%%%%%%%%%%%%%%%%%%%%%
\section{Hydrothermalism in the Deep Ocean}
\label{sec:charting-plumes}
The deep ocean is considered to be any part of the ocean at least \SI{200}{\meter} below sea-level\footnote{Some literature more specifically claims the deep ocean to be at least \SI{1000}{\meter} below the surface. I will use the \SI{200}{\meter} definition, unless otherwise stated.}.
Volumetrically, the deep ocean is \emph{most} of the ocean, and life has been discovered at every depth, including the Challenger Deep, the deepest point on Earth at nearly \SI{110000}{\meter}\autocite{cario2019exploring}\footnote{Interestingly, the average depth of the ocean is approximately \SI{3800}{\meter}; above sea-level, the average height of land is \SI{840}{\meter}.}.
A persistent adage in the ocean sciences is that more is known about the surface of other worlds than about our own deep ocean.
This is poignantly illustrated with the first observation of hydrothermal vents in 1977 at the Galápagos Rift~\autocite{corliss1979submarine}, 8 years after humans walked on the moon for the first time.

Since 1977, hundreds of vents have been discovered and studied around the world\autocite{beaulieu2013authoritative}, and subject of increasingly urgent conversations about characterizing the deep ocean.
Seafloor venting sites, energized by magmatic sources, release fluids between 20-\SI{400}{\celsius} (background deep ocean temperatures are approximately \SI{2}{\celsius}) and imbued with minerals, metals, dissolved gases, and other compounds~\autocite{jannasch1985geomicrobiology, martin2008hydrothermal}.
These warm, nutrient-pumping sites in the deep ocean have created oases for unique micro- and macro-fauna~\autocite{corliss1979submarine}, and the venting fluids, called plumes, can deposit minerals and metals over kilometer (basin) scales\autocite{scholz2019shelf,resing2015basin,le2019hydrothermal}.
Detection and characterization of seafloor hydrothermal venting are critical for understanding fundamental interactions between the deep ocean, its underlying basaltic crust, the deep biosphere, and (bio)geochemical fluxes.

To study hydrothermalism in the deep ocean, rosettes\footnote{Rosettes are often a metal cage with instrumentation and bottles that is attached to a ship via a cable; it can be raised and lowered in the water column with a shipboard winch.}, remotely-operated vehicles (ROVs), human occupied vehicles (HOVs), and autonomous underwater vehicles (AUVs) all equipped with specialized \emph{in situ} instrumentation and often bottles for water sampling, are available.
ROVs and HOVs have enabled detailed study of venting chimneys and diffusive venting fields on the seafloor, literally ``putting eyes'' on the structures and physically interacting with them.
To study plumes generated by vents, AUVs and rosettes can be used to examine the water column.
However, using these technologies to produce detailed studies of plumes is more challenging than studying a vent; plume signal is naturally variable, turbulent, and ephemeral and navigating in the water column is a logistical challenge due to lack of physical features by which to navigate or localize. 
Thus, many detections of a plume tend to be serendipitous in practice.

To leverage these detections, laboratory experiments to model how plumes are expected to manifest in the water column have served a critical role in converting field observations to statements about energetic characteristics, nutrient transport, and overall impact in a basin.
Among the most widely used models that describe hydrothermal plumes are the Morton, Taylor, and Turner (MTT) model for buoyant plumes derived in the 1950s~\autocite{morton1956turbulent}, and the re-derivation specifically for hydrothermal plumes in the late 1980s by Speer and Rona~\autocite{speer1989model}.
These idealized (time-averaged) models describe a roughly two part plume structure composed of a buoyant stem and a neutrally-buoyant layer. 
The buoyant stem is a spatially small expression that describes the fluid that rapidly rises from an originating vent orifice, driven by buoyant forces that result in the difference in density between the warm venting fluid and the ambient seawater.
The neutrally-buoyant\footnote{This can be equivalently styled as non-buoyant.} layer describes the spatially large spread of vent-derived fluids along the isopycnal of equal density with the ambient seawater\footnote{Readers are referred to~\autocite{yoerger2007autonomous} for an illustrative description and figure.}. 
State-of-the-art hydrothermal plume models incorporate time-varying Navier-Stokes models and more complicated fluid structures\footnote{e.g.,~\autocite{lavelle2013turbulent,xu2012deep}}.
Mathematically, these models describe what has been practically well-understood in observational studies: the spatiotemporal distribution of plumes is instantaneously complicated and on small scales (meters, minutes) driven by compounding, chaotic factors that are difficult to calibrate.



\td{Include a nice illustrative figure of plumes?}


Field measurements are used to set the initial conditions or parameters of a numerical model, which is then used in turn to make claims about characteristics of a plume.
This is widely known as solving an \emph{inverse problem}; using observations to find an explanatory set of variables (e.g., initial conditions, numerical parameters).
One of the key challenges of solving an inverse problem with observations in the sciences is that the problem is mathematically ill-posed; that is, it is difficult to find the correct, unique solution due to noise in the observations and the time/space that those observations were taken.
For instance, observations along or near the exact centerline of a plume may be significantly more informative about the vent characteristics\autocite{bangian2022solution} than samples randomly collected throughout a plume structure\autocite{baker1998rise}.
But given that most field measurements are serendipitous, solving the inverse problem to an acceptable level of accuracy requires grappling with \emph{uncertainty}\footnote{Both epistemic and aleatoric. Aleatoric uncertainty in this case comes from the chaotic nature of spatiotemporal distributions (for instance, turbulent flows). Epistemic uncertainty is definitional, as we have uncertainty of the model and noisy observations.}.
Probabilistic representations or analytical models of uncertainty have been used in plume studies to help place confidence intervals over found solutions\autocite{bemis1993geothermal,sohn2019observations}.
In classical studies, uncertainty is computed following an expedition and after all data is available.
In this thesis, an algorithmic extension to the classical uncertainty formalism is extended to enable practical-time inference of plume characteristics while at sea.
By computing a notion of uncertainty while at sea, strategic changes to the science activities and instrument deployments can be undertaken to target collection of more informative samples.


% over the underlying dynamics of a target environment due to the extreme partial observability of point \emph{in situ} measurements in a continuous three-dimensional volume over time influenced by complicated spatiotemporal dynamics of water mixing, tidal advection, and chaotic turbulence.
% The notion of uncertainty has implications for solving both the \emph{inverse} and \emph{forward} problem inherently posed in hydrothermal plume charting.
% Discovering the underlying model from a set of data, the inverse problem, is known to be ill-posed in spatiotemporal systems CITE, suggesting that the information content of every sample has an outsized effect on the accuracy of recovered dynamics.
% With a known dynamics model, forward simulating a model with uncertainty placed over initial conditions or descriptive parameters requires propagating that uncertainty forward.
% In time-dependent models, even modest amounts of uncertainty can accumulate quickly over short horizons CITE, yielding essentially uninformative predictive state estimates for planning.
% A principled treatment of uncertainty, particularly the identification of \emph{useful} uncertainty for planning, is necessary. 
% WORDS ON HOW PEOPLE HAVE THOUGHT ABOUT UNCERTAINTY IN HYDROTHERMAL SYSTEMS


% Questions related to mineral and metal deposition from plumes highlight the acute scientific need to better study not just hydrothermal vents themselves, but the plume structures that are produced by them which are the primary drivers of nutrient injection in the deep ocean biosphere.
% Scientific studies that benefit from directly charting the spatiotemporal structure of a plume include those which characterize the transport and deposition of particulates, the distribution and transport of biological materials (including microbes) in the deep ocean, and field-based statistical modeling of turbulent water mass mixing.


%%%%%%%%%%%%%%%%%%%%%%%%
% Autonomy Challenges
%%%%%%%%%%%%%%%%%%%%%%%%
\section{Challenges for Intelligent Autonomy}
For geochemical studies in the water column, AUVs are uniquely well-positioned to advance long-term monitoring of and exploration in mesoscale\footnote{Tens of meters to several kilometers} deep ocean environments during \emph{in situ} expeditions.
Autonomy for a robotic system can fall on an ``agency'' spectrum: a robot with full agency can make adjustments to its own behavior; a robot with no agency can execute a pre-determined set of tasks without supervision, but cannot change the tasks while performing.
Increasingly, AUVs are being developed for deep sea research\autocite{kaiser2016design,yuh2000design,okamoto2019visual,maki2014auv}, but their autonomous capabilities are typically limited to executing predetermined hand-designed trajectories such as uniform coverage lawnmowing patterns~\autocite{camilli2010tracking}, falling in the category of robots without agency.
This restriction is often applied in order for trajectories to be rigorously checked by engineering teams prior to execution as an operational policy of risk reduction, and for ease of supervision during execution.

Operating without agency necessarily restricts the class of phenomena that can be effectively studied by expeditionary robots used in the science fleet today.
Using non-adaptive surveying strategies, spatiotemporal distributions can be severely under-sampled or missed completely\autocite{flaspohler2019information, preston2019adaptive}.
This can be mitigated when the underlying model of the spatiotemporal dynamics is known, but for reasons explained in the previous section, in field settings a dynamics model must be estimated from data.
This estimation process is challenging, as data that can be collected in real field trials tends to be noisy and extremely \emph{partially observable}---that is, the observations are only at point locations in time and space, and may be indirect measurements of a desired field of interest. 
Consequently, classical data-driven modeling techniques used in robotics to describe the environment in which a robot finds itself, such as Guassian Processes\autocite{Rasmussen2004}, neural networks\autocite{cohn1994neural,wang2017predrnn}, or particle representations\autocite{Silver2010}, would require many observations to generalize to a useful model for planning; a luxury that, in the field, is typically not afforded due to limited opportunities for deployments and finite expedition timelines.


\td{Add illustration of naive versus strategic lawnmower placement.}


Access to numerical models of plume dynamics stands to significantly relieve the burden on data alone to recover a descriptive model of the target environment during an expedition.
Scientific machine learning (SML), an emerging subfield of machine learning, has shown that leveraging numerical scientific models of physical principles within classically data-driven frameworks\autocite{raissi2019physics, sapsis2009dynamically, mohan2019compressed,raissi2018numerical,kulkarni2019advection, brunton2016discovering,jiahao2020learning} can improve the data efficiency of a learner and the overall quality of the model that is uncovered.
To extend SML algorithmic theory to field settings, in which computational time is limited and field observations are corrupted with noise (and significantly partially observable), requires careful selection of both the underlying numerical model and the learning framework that wraps it.
Foundational principles (e.g., conservation of mass, non-divergence) 



%%% Starting here, needs some editing attention
To extend mathematical models and this algorithmic theory to field settings, in which computational time is severely limited and field observations are corrupted via noise, the foundational\footnote{And implicitly more straightforward or simpler to compute} principles of an assumed underlying dynamical model may still be an informative basis on which to learn.
For instance, the concepts of non-divergence in flow fields and conservation of mass, and simplified models of time-averaged behaviors, can serve as an inductive bias\footnote{The term \emph{inductive bias} has a mixed connotation in learning. Designing an inductive bias will, definitionally, bias a learner towards the assumptions for unseen data. This is wonderful for an environmental context in which we're guaranteed to seeing things that were not in the original training data and for which grounded principles imply underlying structure, but in other contexts (particularly social and cultural), inductive bias in learned models may reflect and enforce systemically inappropriate or unconscious assumptions.} for probabilistic data-driven models.
Using \emph{physically-informed structure} reduces the burden on the data alone for recovery of a dynamics model and the use of idealized analytical models can be considered a principled technique for identifying a reduced-basis for inference (rather than finding a latent space via the data directly).
As some in robotics may say, however, there is no free lunch.
The use of physically-informed structure trades sample efficiency for increased computational cost at deployment time (in comparison to purely data-driven techniques).
However, this trade-off can be tuned for the underlying models available, the sampling task at hand, and the operational logistics during field opportunities.

In a field setting, embedding scientific knowledge into probabilistic models for robot or human decision-making requires additional, necessary algorithmic infrastructure. 
Surrounding a model there must be an observation model and a decision-maker/planner.
The observations that a robot collects must correspond to the internal representation used by the robot (it's belief) with the science models embedded.
As scientific observations are typically taken by heterogeneous sensors with different operating principles\footnote{And therefore may measure different phenomenon with complicated interrelationships.}, the choice of observation model is not straightforward.
While it is unreasonable to assume that a roboticist become a domain expert in a particular scientific field in order to plan useful trajectories, familiarity with the forms, limitations, and working principles of critical science sensing infrastructure is necessary in order to make advances in modeling and planning.
It's a core tenant of this work that the development of expeditionary robotics cannot happen in a vacuum; collaboration with scientists who will ultimately use this technology must be undertaken.

The decision-maker can be either a person (or a team of people), or an automated process (such as the robot itself). 
To enable a decision-maker, the output of the probabilistic model must be interpretable by some means---either semantically for human-readability, or technically for automated optimization via the specification of a reward function.
In any case, a decision-maker must work under often complicated and changing constraints in the field, which can disrupt or inhibit the planning process.
For instance, in the case in which trajectories need to be computed hours before a planned deployment window of a robot, if that deployment window changes due to e.g., weather, the plans themselves must either be agnostic or flexible enough to adapt in response to the change within the updated timeframe set by engineering teams operating the vehicles.
As computing universally good plans under any change of conditions is intractable, we must rely on more heuristic opportunities to design easily interpreted and modifiable plans (potentially by hand), to ultimately avoid a multi-hour long replanning-rechecking procedure.

This thesis presents the case for embedding scientific knowledge into the observation model (\emph{perception}), belief representation (\emph{prediction}), and decision-making framework (\emph{planning}) of robotic operations in a science expedition to tackle core challenges related to vehicle nonadaptivity, partial observability in spatiotemporal domains, and operational restrictions. 
To ground the work, the specific expeditionary problem of deep-sea hydrothermal plume charting is closely analyzed, and a field deployment with autonomous underwater vehicle (AUV) \Sentry in the Guaymas Basin, Gulf of California is used to demonstrate the algorithmic contributions to be presented.
Deep-sea hydrothermalism is the thermal and geochemical circulation of seawater driven by magmatic activity.
Characterizing interaction between the magmatic layer, basaltic crust, and deep ocean biosphere is of contemporary interest with this thesis, as concerted efforts to pursue deep sea mining (for recovery of rare earth materials and metals) and deep carbon sequestration are hotly developing fields.
Hydrothermalism plays a critical role in these processes, as a nutrient (and metals) pump, indicator for geological activity, and sustainer of unique ecosystems.
To chart hydrothermal plumes means to survey the structure of the warm, nutrient-rich water that is ejected from vents on the seafloor.
As plumes are subject to instantaneous turbulence, advective crossflows, and complex mixing, predicting the location of a plume from point observations poses a considerable challenge for planning trajectories for a nonadaptive robotic platform.
 
% The problem of discovering a useful forward simulator for spatiotemporal is not impossible, however.
% Decades of research have been dedicated to the recovery of environmental models by experimental trials and mathematical reasoning by scientists.
% For instance, the Navier-Stokes equations, which describe the motion of viscous fluids, are the philosophical, mathematical, and experimental results of Leonhard Euler~\autocite{euler1757principes}, Claude-Louis Navier~\autocite{navier1822lois}, and George Gabriel Stokes~\autocite{stokes1851effect} stretching from 1757 to 1851. 
% Each mathematician started from the knowledge of their predecessor(s), and extended the sophistication of the model in turn.
% This is a natural scientific process, and it is a process that is well suited to inspire algorithmic adaptation.


%%%%%%%%%%%%%%
% Contributions
%%%%%%%%%%%%%%%
\section{Contributions}
The contributions of this thesis are...

In this thesis, a novel idealized model for buoyant plume rise in crossflow (i.e., current/advective forces) is applied to the hydrothermal plume setting\autocite{tohidi2016highly} and coupled with a unique uncertainty representation framework to update the model from realistic field observations.
In \cref{chap:phortex}, the model and belief representation is presented, and demonstrates 

%%%%%%%%%%%%%%%%%%%%%%%%%%%%%%%%%%
% Thesis Overview
%%%%%%%%%%%%%%%%%%%%%%%%%%%%%%%%%%
\section{Thesis Overview}
The contributions of this thesis are\dots

The remainder of this thesis is organized as follows\dots