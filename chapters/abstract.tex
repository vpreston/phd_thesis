\begin{center}
    {\large \@title} \\
    \emph{\footnotesize by} \\
    \@author \\
    \end{center}
    
    \vspace{-1em}
    
    \begin{center}
    \begin{singlespace}
    {\parindent0pt
    \small
    Submitted to the Department of Aeronautics and Astronautics and the Joint Program in Applied Ocean Science \& Engineering on \@date ~in partial fulfillment of the requirements for the degree of Doctor of Philosophy}
    \end{singlespace}
    \end{center}
    
    \begin{singlespace}
    {\parindent0pt 
        {\large \textsc{Abstract}} \\ %less than 200 words for WHOI (350 for MIT)
    
        % Context:

        % Importance:

        % How:

        % Challenge:

        % Intuition/Thesis Statement:

        % How Intuition was Applied:

        % Results:

        % Impact:


        Making measurements of the natural world and building accurate models is crucial to understanding our environment and the spatiotemporal processes that drive our climate. 
        Autonomous robots are especially well-suited to gathering those measurements efficiently. 
        However, to collect useful observations of unknown, partially-observed spatiotemporal distributions for scientific inquiry requires accurately perceiving a phenomenon of interest, predicting how it will evolve in time, and planning effective sampling trajectories, potentially under severe operational constraints. \\

        % This talk will present the problem of deep-sea hydrothermal plume charting with an autonomous underwater vehicle (AUV) as part of an oceanographic research expedition. In this setting, the AUV, by matter of policy and ability, is operationally restricted to executing pre-set, fixed trajectory patterns (i.e., the robot’s behavior cannot respond to observations). Yet, the AUV is tasked with densely surveying the spatiotemporal structure of an a priori unknown moving geochemical field, the state of which is only partially-observable and subject to unseen forces (e.g., advective current, diffusive mixing). To enable this capability for an AUV, I will describe a deployment-by-deployment iterative learning and trajectory optimization process, PHORTEX: Physically-informed Operational Robotic Trajectories for Expeditions. In this process we learn to predict the state of the spatiotemporal plume from a history of partial scientific measurements by embedding idealized scientific priors (e.g., analytical models of plume rise) in a Bayesian inference framework. I will present results from 2021 field experiments at a deep sea (2000 m) hydrothermal vent field in the Gulf of California using AUV Sentry.\\
    }
    
        \noindent Thesis Supervisor: Nicholas Roy \\
        \noindent Title: Bisplinghoff Professor of Aeronautics and Astronautics, MIT \\

        \noindent Thesis Supervisor: Anna Michel \\
        \noindent Title: Associate Scientist with Tenure, Applied Ocean Physics and Engineering, WHOI

    \end{singlespace}
    
    \newpage
    \null
    \thispagestyle{empty}
    \newpage

