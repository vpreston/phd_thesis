\begin{center}
    {\large \@title} \\
    \emph{\footnotesize by} \\
    \@author \\
    \end{center}
    
    \vspace{-1.5em}
    
    \begin{center}
    \begin{singlespace}
    {\parindent0pt
    \small
    Submitted to the Department of Aeronautics and Astronautics and the Joint Program in Applied Ocean Science \& Engineering on \@date ~in partial fulfillment of the requirements for the degree of Doctor of Philosophy}
    \end{singlespace}
    \end{center}
    
    \begin{singlespace}
    {\parindent0pt 
        {\large \textsc{Abstract}} \\ %350 word limit

       To better understand our ocean would be to better characterize the largest habitable biosphere on planet Earth, better quantify the geochemical processes that control Earth's climate, and better regulate natural resources stored in its depths. \emph{Expeditionary science} is the art of collecting \emph{in situ} observations of an environment to build approximate models of underlying properties that move us towards better understanding. Robotic platforms are a critical technology for collecting observations of the ocean. Depth-capable autonomous underwater vehicles (AUVs) are commonly used to build static maps of the seafloor by executing pre-programmed surveys. However, there is growing urgency to generate rich data products of \emph{spatiotemporal distributions} that characterize the physics and chemistry of the deep ocean biogeosphere. In this thesis, the problem of charting deep sea hydrothermal plumes with depth-capable AUVs is investigated. Effectively collecting samples of geochemical plumes using the operationally preferred strategy of pre-specifying surveys requires access to a dynamics model of the advective currents, bathymetric updrafts, and turbulent mixing at a hydrothermal site. In practice, however, access to this information is unavailable, imperfect, or only partially known, and so a model of plume dynamics must be inferred from observations and subsequently leveraged to improve future sampling performance. As most \emph{in situ} scientific instruments yield point-measurements, considerable uncertainty is placed over the form of the dynamics in purely data-driven solutions. Challenges related to \emph{planning under uncertainty} for geochemical surveys in the deep ocean are addressed in this thesis by embedding scientific knowledge as a strong inductive prior for tractable model learning and decision-making. Algorithmic contributions of this thesis show how plumes can be \emph{perceived} from field data, their fate \emph{predicted} far into the future (e.g., multiple days), and informative fixed trajectories \emph{planned} which place an AUV in the right place at the right time. Alongside algorithmic analysis, scientific assessment of observational data collected with AUV \Sentry during field trials in the Guaymas Basin, Gulf of California are interwoven, and demonstrate how intelligent perception, prediction, and planning enables novel insights about hydrothermal plumes.\\ 
       
       %This thesis presents the first study to quantify the extent of hydrothermal expressions in the Northern Guaymas Basin, the first iterative deployment of an AUV for deep sea plume charting, and novel representation to study plume formation under advective crossflow useful for planning and scientific analysis.  \\
       
       
       %Challenges related to \emph{planning under uncertainty} for geochemical surveys in the deep ocean are addressed in this thesis by embedding scientific knowledge as a strong inductive prior for tractable model learning and decision-making. Algorithmic contributions of this thesis show how plumes can be \emph{perceived} from field data, their fate \emph{predicted} far into the future (e.g., multiple days), and informative fixed trajectories \emph{planned} which place an AUV in the right place at the right time. Alongside discussion of the technical implications of embedding scientific knowledge into algorithmic representations, this thesis will also present scientific analyses that enhance our understanding of hydrothermalism in the Guaymas Basin from data collected by AUV \Sentry in November 2021 using intelligent perception, prediction, and planning frameworks.\\ 
        
        
        % To better understand our ocean would be to better describe one of the largest ecosystems on planet Earth, better quantify the geochemical processes that control Earth's climate, and better regulate natural resources stored in its depths. \emph{Expeditionary science} is the art of collecting \emph{in situ} observations in a target environment so as to build approximate models of underlying environmental properties that move us towards better understanding. By the nature of ocean research, robotic platforms are a dominant technology driving \emph{in situ} observations for expeditionary science in the deep sea. Historically, depth-capable autonomous underwater vehicles (AUVs) have been used to perform simple surveying tasks in the deep ocean (e.g., bathymetric mapping) that can be fully pre-specified by human operators and which yield high-quality reconstructions of static fields. However, there is growing urgency to utilize robots to generate rich data products of spatiotemporal distributions, such as geochemical plumes. To effectively collect samples from spatiotemporal distributions using the operationally preferred strategy of pre-specifying trajectories requires access to a dynamics model of the environment to optimally place surveys for a given task. Unfortunately, access to the true dynamics model of an underlying distribution of a target environment is generally unattainable, and so the model must be inferred from observations that are feasible to collect using classical strategies, and then leveraged to improve future performance. As most \emph{in situ} scientific instruments yield point-measurements, considerable uncertainty is placed over the form of the dynamics due to extreme partial observability. In this thesis, planning under uncertainty for robotic expeditionary science is addressed by embedding scientific knowledge as a strong inductive prior for tractable model learning. Applied illustratively to the problem of charting the spatiotemporal structure of deep-sea hydrothermal plumes, algorithmic contributions are presented which show how to \emph{perceive} plumes, \emph{predict} their state far into the future (e.g., multiple days), and \emph{plan} informative fixed trajectories. Each chapter will discuss the technical and scientific implications of embedding scientific knowledge into algorithmic representations, grounding the discussion with results from field trials completed in November 2021 with AUV \Sentry at the Guaymas Basin, Gulf of California.\\ 
        
        
        % Context:
        % Transient, dynamic phenomena are of interest in many of the disciplines of observational science. \textit{Expeditionary science} encapsulates the observational sciences that require \textit{in situ} sample collection of environmental phenomena.
        % % Importance:
        % Taking physical measurements of the natural world and building accurate models from those observations is crucial to understanding a phenomenon and the spatiotemporal processes that drive it. Autonomous robots are especially well-suited to gathering those measurements efficiently.
        % % Challenge:
        % However, to collect useful observations of unknown, partially-observed spatiotemporal distributions for scientific inquiry requires accurately perceiving a phenomenon of interest, predicting how it will evolve in time, and planning effective sampling trajectories to put the robot in the right place at the right time, potentially under severe operational constraints.
        % % Intuition/Thesis Statement:
        % % In this thesis, the core challenges of \emph{extreme partial observability} and \emph{limited adaptive intelligence} are overcome by introducing methods embedded with domain scientific expertise to gain computational tractability with operating in the field.
        % Key to addressing these challenges is uncovering the underlying physics which describe phenomenon evolution.
        % This thesis establishes algorithmic contributions for computing tractable, probabilistic  solutions to inverse problems for long-horizon forward simulation of a target environment.
        % To do so, idealized scientific summaries of a given target environment for a particular task are formulated and embedded as strong inductive priors into learning frameworks suitable for use in robotic informative path planning.
        % % How Intuition was Applied:
        % To ground the algorithmic discussion, the problem of charting deep-sea hydrothermal plumes with a robot restricted to executing non-adaptive behaviors is used as a scientific and technical context for development.
        % As such, this thesis presents novel methods for detecting hydrothermal plumes from heterogeneous instruments, models for utilizing those observations to uncover and forward simulate plume dynamics, and a trajectory optimization strategy suitable for state-of-the-art expeditionary robots in deep-sea oceanographic research.
        % % Results:
        % Each chapter will discuss the technical and scientific implications of the proposed methods, drawing on results from a real field trial for mapping hydrothermal plumes with autonomous underwater vehicle (AUV) \emph{Sentry} in the Gulf of California at the Guaymas Basin in 2021.
        % % Impact:
        % The outcome of this thesis is an illustrative example of an autonomy system which extends the capabilities of expeditionary robots towards previously unattainable queries about dynamic phenomena.\\
    }
    
        \noindent Thesis Supervisor: Nicholas Roy \\
        \noindent Title: Bisplinghoff Professor of Aeronautics and Astronautics, MIT \\

        \noindent Thesis Supervisor: Anna Michel \\
        \noindent Title: Associate Scientist with Tenure, Applied Ocean Physics and Engineering, WHOI

    \end{singlespace}
    
    \newpage
    \null
    \thispagestyle{empty}
    \newpage

% To enable modern expeditionary robots to perform complex tasks under limited self-agency requires a tractable means of discovering the underlying physics of a target phenomenon and usefully forward simulating it over long-horizons for trajectory planning. 
% Solving inverse problems of natural environments is computationally taxing, and impractical for use in field applications.
% Moreover, discovering physics from scratch using data, let alone noisy, partial observations, is also computationally taxing.
% But the intuition here is that we need to be able to do something useful, so we embed science as an inductive bias into data-driven models which perform simplified dynamical forward simulations.