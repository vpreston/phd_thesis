\chapter{Introduction}

The environmental sciences are a multidisciplinary endeavor to understand the Earth, its ecosystems, and its processes, for which \emph{in situ} observational studies or \emph{expeditions} serve as the foundation on which scientific discovery and model development is predicated. Robots are uniquely well-positioned to advance long-term monitoring of and exploration in meso-scale planetary environments through autonomous expeditions. By virtue of their form, robots can be used in extreme places (e.g., deep sea), dangerous scenarios (e.g., edge of calving ice sheets), or long-term missions (e.g., Mars exploration). Increasingly, robotic platforms are being developed for scientific expeditions, but their autonomous capabilities are typically limited to predetermined hand-designed trajectories (e.g., uniform coverage lawnmowers \cite{camilli2010tracking}). This significantly restricts the class of phenomena that can be effectively studied by expeditionary scientific robots. For instance, spatiotemporal distributions---deep sea hydrothermal plumes, algal blooms, weather cells---can be severely under-sampled or missed using these strategies \cite{flaspohler2019information}. Given the ubiquity of these spatiotemporal phenomena and the cost of scientific field operations, it is critical to improve the efficacy of robots as autonomous scientific tools.

This thesis represents\dots
