\chapter{Operations at Sea}
%Preface that the intent of this chapter is to highlight practical engineering necessities and opportunities for deep sea research.

\begin{center}
    \begin{minipage}{0.5\textwidth}
      \begin{small}
        How inappropriate to call this planet Earth when it is quite clearly Ocean.\\ \emph{Arthur C. Clarke}
      \end{small}
    \end{minipage}
    \vspace{0.5cm}
\end{center}

Field robotics is a subfield of research devoted to enabling sophisticated robotic autonomy or control directly within the environmental and task contexts for which they were designed.
Being a field roboticist means developing sophistication with practicalities in mind.
Practicalities in the case of launching expediitonary robots may include the location and nature of the deployment site, the support infrastructure necessary for robot deployment and human life-support, 

 
\section{Challenges in the Deep Ocean}
No GPS, no satellite, only acoustics, very few observatories, etc.

\section{Overview of Science Teams and Responsibilities}
Establish how computer scientists fit on a ship.

\section{Data Infrastructure on a Vessel}
Propose live-streaming\dots

\section{Taking Ground Truth Measurements}
Basically impossible, some things more than others.

\subsection{Water Column Standards}
Profiles gathered

\subsection{Hydrothermal Vents}

\subsubsection{Geochemical Measurements}
Jason wand/standard equipment

\subsubsection{Physical Measurements}
Fluid exit velocity, PIV system

\subsection{Crossflow}
Tiltmeters
