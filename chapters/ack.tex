\chapter*{Acknowledgments}
% Environmental science was something introduced to me as a child with field trips to the Smithsonian Environmental Research Center (SERC) on the Chesapeake Bay in central Maryland. 
% There, they would teach us about the role of marshes as a great ``buffer'' for pollution, take us out seining and show us how to identify all the little critters that got caught up in the nets, and let us visit field stations where water quality was monitored.
% These field trips were part of a larger effort to engage children, their families, and their communities in feeling connected and responsible for the Bay's health.
% It is immensely satisfying now to be able to engage with environmental science as a roboticist

Research is a team sport, and I have profound appreciation for my teammates, coaches, and enthusiastic supporters that have made this work possible.

Since I started graduate school, I have had the great fortune of collaborating closely with Genevieve Flaspohler, with whom I've talked for hours about algorithms, life, and expeditionary science. I'm so grateful for her significant contributions to this work, as well as her friendship. Claudia Cenedese, Dan Fornari, Pete Girguis, Scott Wankel, Chris German, Michael Jakuba, Guangyu Xu, John Fisher, Thibaut Barreyre, and Xubo Zhang provided invaluable insights about hydrothermalism, the challenges of studying plumes in the deep ocean, and planning under uncertainty; from providing technical mentorship to datasets to sensors to field work opportunities, I have been truly humbled by their expertise, willingness to hear out my quirky ideas, and enthusiasm for robotic tools. Dan Yang and Valentin Peretroukhin participated in developing some of the first prototypes of the algorithmic contributions in this thesis, and their insights and assistance the months before going to sea was critical for the success of the field campaign. Warmest thanks to the AUV \Sentry team, and especially Sean Kelley, Zac Berkowitz, Justin Fujii, Amanda Sutherland, Joe Garcia, Stefano Suman, and Isaac Vandor, all of whom provided great technical and practical insight into AUV operations, and allowed us to try something a little different on RR2107. I fondly appreciate the camaraderie of the science party and ROV \emph{JASON} team on RR2107, who made my watch shift in the wee hours of the morning fun, explained their projects to me with great patience, and always made sure to remind me when meal times were. The captain and crew on the R/V \emph{Revelle}, who shared their home with me for a few weeks in November, cannot be commended enough for their professionalism and friendliness, curiosity about the science, and willingness to pull off that one final transect experiment when sailing home was just within reach.  

Anna and Nick have been key players in all of my doctoral work. It has been a profound privilege to work with two professionals in different fields and getting their unique perspectives on my research; their shared enthusiasm for the work is plain in the ability for this thesis to span multiple disciplines. Anna has opened so many doors for me in the ocean sciences, not least of which have been opportunities to go to sea, and I've been undoubtedly hooked by ocean research and challenges as a result. I have deeply appreciated our conversations about inclusion in the ocean sciences, her attention to my professional development, and her insights on the next frontier of ocean technology and robotics. When my studies at MIT began, Nick was willing to give me some desk space while I was on campus; I am so grateful that he took some interest in my research and has since been an incredible influence in how I think about problem solving. Nick's ability to give high quality feedback, and willingness to receive feedback in return, has made me feel both professionally and personally valued, which helped me stay engaged when the going got tough. Both Anna and Nick have built incredible labs around them, and I am so grateful to my colleagues in the Robust Robotics Group and the Chemical Sensor Lab, folks who I also count among my friends. I especially want to express appreciation for Beckett Colson, Chris Bradley, Michael Noseworthy, and Martina Stadler for their friendship and feedback on my work throughout our studies together. As part of the research experience in Nick and Anna's groups, I've had the good luck of advising incredible undergraduate students who helped me develop as a mentor and researcher. I am so appreciative of their patience and for giving me the opportunity to work on some really cool projects that otherwise would have never gotten off the ground. 

My committee members, Youssef Marzouk and Adam Soule, have been thoughtful contributors to this thesis, and have devoted considerable time towards thinking about my research. I am deeply thankful for their insights and expertise. Thanks also go to Kwesi Rutledge and Valentin Peretrouhkin who are external readers for this work, and are folks whose work and way of thinking about research and research communities I greatly admire.

Finally, a special thanks to my family and to my partner, Bill. My parents shaped me into the independent person I am today, and have modeled hard work, determination, and a get-it-done attitude all my life. Along with my brother, I am so grateful for their enthusiasm and support. While Bill's suggestion for an acronym of the binary sensor filter presented in this thesis ultimately didn't make the cut (it was PHINDS: Procedural Heuristic Integration of Nautical Disparate Sensors, for the curious), I think it's indicative of how much of his own time and energy he's invested in supporting me and becoming familiar with this work. Bill has been a wonderful life partner and research sounding board throughout my graduate school experience, and I'm sincerely grateful for him.

\begin{flushright}
Thank you all,\\ 
Victoria
\end{flushright}

\vspace{1em}

\noindent Financial support for my research was provided by the National Defense Graduate Fellowship Program and the MIT Martin Family Society of Fellows for Sustainability. Research activities for the RR2107 cruise were funded by NSF OCE OTIC \#1842053, a WHOI Innovation Technology Award, NOAA Ocean Exploration \#NA18OAR0110354, and Schmidt Marine Technology Partners Award \#G-21-62431. 
