\chapter{Final Thoughts}
\label{chap:conclusion}

\begin{center}
    \begin{minipage}{0.7\textwidth}
      \begin{small}
        I believe that every engineer has a responsibility to make the world a better place. We are gifted with an amazing power to take people's wishes and make them a reality.\\ \emph{Ayanna Howard}
      \end{small}
    \end{minipage}
    \vspace{0.7cm}
\end{center}


In November 2021, AUV \Sentry was used to chart hydrothermal plumes in the first field demonstration of a physically-informed autonomous mission planner for deep sea geochemical surveys.
This thesis presents the algorithmic underpinnings and preliminary scientific results.
Fundamental to the research program presented here is the notion that robots and robotic algorithms are tools that are best selected when the context of a problem domain is fully defined. Expeditionary science, the art of collecting useful \emph{in situ} observations of natural phenomena, provides a context that requires tools that can parse complex, sparse observations into interpretable summaries of spatiotemporal distributions, reason over uncertainty to generate environmental forecasts, and leverage sophisticated sensing platforms to the fullest of their abilities.

Several tools are presented in this thesis. In \cref{chap:afar}, a set of anomaly detection and temporal analysis methods are presented that succinctly summarize complex, heterogeneous data from \emph{in situ} geochemical sensors to assist a science party in identifying hydrothermal expressions. Using these tools demonstrated that hydrothermalism could be confidently identified far from a generating vent in the Guaymas Basin (4-\SI{7}{\kilo\meter}) in the first study to quantify the extent of hydrothermal intrusion in the water column in the Northern Basin. These tools were later adapted in \cref{chap:field_results} to create an automatic hydrothermalism detector for use in a robotic decision-making framework.

That framework, presented in \cref{chap:phortex} and demonstrated in the field in \cref{chap:field_results}, consisted of a belief representation, \PHUMES: \phumes, and an algorithmic decision-maker, \PHORTEX: \phortex. \PHUMES leveraged a novel time-averaged model of buoyant plumes in stratified environments with crossflow to generate forecasts over the location of plume fluid from a history of observations. The formulation of \PHUMES positioned the belief representation as a universal data aggregator not just for \Sentry operations, but all other science activities in the water column, enabling \PHUMES to produce forecasts with predictive power several days into the future. By virtue of the choice of analytical model, \PHUMES was able to examine the structure of neutrally-buoyant plumes observed in the Basin, and generate novel hypotheses about their formation and characteristics that would otherwise be impossible to resolve using classical models of buoyant rise in the deep ocean.

With \PHUMES in hand, \PHORTEX generated trajectories that enabled AUV \Sentry to track moving hydrothermal plumes without adaptive capabilities. Using chained lawnmower trajectory patterns, \PHORTEX trajectories ``fanned out'' over the course of a multi-hour dive to effectively re-encounter and survey plume structures, collecting spatiotemporally diverse datasets that sharply contrast with human-designed surveys which tend to bias detections of hydrothermalism to within a small distance of a generating vent. With operational changes made on the fly while at sea, the ability for \PHORTEX to design trajectories that could robustly sample a plume demonstrates its utility for future field campaigns. 

In \cref{chap:future}, this thesis makes a direct appeal for consideration by the robotics community of expeditionary science contexts, asserting that expeditionary science provides interesting constraints and requirements that challenge the state-of-the-art tools currently available today. Moreover, roboticists are uniquely equipped to impact the strategic and urgent vision for improving scientific understanding of the Earth, the ocean, and the processes within them. Several possible project settings are defined in detail, presenting both algorithmic and contextual advances ripe for development. Uniting all of these projects is a persistent need for belief representation tools that can usefully describe spatiotemporal distributions which can only be partially observed. The key insight of this thesis is that scientific models can be embedded as principled priors and bias in learning frameworks. Not only does embedding knowledge allow for tractable recovery of spatiotemporal dynamics, but the belief representations themselves can directly support scientific inquiry by virtue of being interpretable, grounded by accepted principles, and familiar to scientist stakeholders. For advances in scientific machine learning, stochastic model learning, and physically-informed belief representations to make an impact, it is imperative that the systems in which these algorithms are to be deployed---the environments, people, platforms, and operations---be considered, and ideally involved through collaborative projects.

In closing, the research in this thesis is the culmination of collaborative efforts between scientists and engineers to do something that has never been done before, but which will hopefully be done hundreds of times more, and better, into the future. It is a distinct privilege of a field roboticist to have the ability to stretch between academic silos and work with diverse teams to make sophisticated autonomy while getting their boots wet on the deck of a ship. It also comes with a distinct responsibility to prove that their tools not only work, but that they address the need presented by sponsors/stakeholders, collaborators, and communities. Showing that a need is addressed is not necessarily as straightforward as showing the some cost function is minimized; it requires following up, pursuing the science, and creating adjustments that can make all the difference in a tool being used again. Being a meaningful part of a team makes engaging with the process of demonstrating impact natural and not onerous; necessary and not frivolous. Responsibility to one another and towards being stewards of Earth will be potent drivers for the development of robotic expeditionary science in the years to come.

